\documentclass{article}
\usepackage{amsmath}
\usepackage{amssymb}
\usepackage{amsfonts}
\usepackage{mathtools}
\usepackage{mathrsfs}
\usepackage{amsthm}
\usepackage{bbm}
\usepackage{graphicx}
\usepackage{pgfplots}
\usepackage{tikz}
\usepackage{geometry}
\geometry{margin=1in}

\title{Differential Equations - MATH246 Discussion Notes}
\author{Tom Mitchell}
\date{Han - Fall 2024}

\begin{document}

\maketitle

\section*{Class Information}

\subsection*{Grading}
\begin{itemize}
    \item Matlab assignments — 18\% (6\% each)
    \item Quizzes (drop two lowest) — 17\%
    \item Two best in-class exams — 17\% each
    \item Worst in-class exam — 8\%
    \item Final exam — 23\%
\end{itemize}

\subsection*{Office Hours}
\begin{itemize}
    \item Monday: 2:00 PM - 3:00 PM (in person, Kirwin 2400)
    \item Tuesday: 1:15 PM - 2:30 PM (in person, Kirwin 2400)
    \item TBA: Zoom (online)
\end{itemize}

\subsection*{Exams}
\begin{itemize}
    \item 3 midterms and a final exam
\end{itemize}

\section*{Discussion \#1}

\textbf{Problem I.1.(1).}

Given the explicit differential equation: \(\frac{dy}{dt} = f(t)\)

The solution is given by: 
\[
y(t) = \int f(s) \, ds
\]

For the specific case: 
\[
\frac{dy}{dt} = 2\sin(t)\cos(t)
\]

\begin{enumerate}
    \item[(a)] Show that \(y(t) = \sin^2(t)\) is a solution to this equation for every \(t \in (-\infty, \infty)\).
    
    \item[(b)] Give the general solution to this equation.
\end{enumerate}

\textbf{Solution:}

\begin{enumerate}
    \item[(a)] We start by differentiating \(y(t) = \sin^2(t)\):
    \[
    \frac{d}{dt}\left(\sin^2(t)\right) = 2\sin(t)\cos(t)
    \]
    Hence, \(y(t) = \sin^2(t)\) is indeed a solution to the differential equation \(\frac{dy}{dt} = 2\sin(t)\cos(t)\) for every \(t \in (-\infty, \infty)\).

    \item[(b)] To find the general solution, integrate the right-hand side of the differential equation:
    \[
    y(t) = \int 2\sin(s)\cos(s) \, ds
    \]
    Using the trigonometric identity \(2\sin(s)\cos(s) = \sin(2s)\), we get:
    \[
    y(t) = \int \sin(2s) \, ds = -\frac{1}{2}\cos(2s) + C
    \]
    Thus, the general solution is:
    \[
    y(t) = \sin^2(t) + C
    \]
    where \(C\) is the constant of integration.
\end{enumerate}


\subsection*{Linear Normal Form}

The general linear normal form of a first-order differential equation is:

\[
\frac{dy}{dt} + a(t)y = f(t)
\]

where \(a(t)\) and \(f(t)\) are functions of \(t\).

To solve this equation, we first define \(A(t)\) as:

\[
A(t) = \int a(s) \, ds
\]

The integrating factor is given by:

\[
\text{Integrating factor} = e^{A(t)}
\]

Multiplying the entire differential equation by the integrating factor:

\[
e^{A(t)} \frac{dy}{dt} + e^{A(t)} a(t) y = e^{A(t)} f(t)
\]

Recognizing that the left-hand side is the derivative of \(e^{A(t)}y\), we have:

\[
\frac{d}{dt} \left(e^{A(t)} y \right) = e^{A(t)} f(t)
\]

Integrating both sides with respect to \(t\):

\[
e^{A(t)} y = \int e^{A(s)} f(s) \, ds
\]

Finally, solving for \(y(t)\):

\[
y(t) = e^{-A(t)} \int e^{A(s)} f(s) \, ds
\]

\textbf{Problem I.2.(1).}

Determine which of the following equations are linear. For the ones that aren't, explain where the problem lies. For the ones that are, put them in linear normal form if they aren't already.

\begin{enumerate}
    \item[(a)] \(y' = t y - 7\)
    \item[(b)] \(t y' + y^2 = e^t\)
    \item[(c)] \(y y' = 5t - \sin(y)\)
    \item[(d)] \(\frac{y'}{t} + \tan(t) y = 1\)
    \item[(e)] \(y' + \frac{t}{y} = t^2\)
    \item[(f)] \(\sin(t) y' - \frac{\log(t^2) + \frac{y}{t}}{1 + e^{2t}} = y'\)
\end{enumerate}

\subsection*{Solution}

\begin{enumerate}
    \item[(a)] \(\mathbf{y' = t y - 7}\) \\
    Yes, this equation is linear. It is already in linear normal form:
    \[
    \frac{dy}{dt} - t y = -7
    \]

    \item[(b)] \(\mathbf{t y' + y^2 = e^t}\) \\
    No, this equation is not linear. The nonlinearity arises from the \(y^2\) term.

    \item[(c)] \(\mathbf{y y' = 5t - \sin(y)}\) \\
    No, this equation is not linear. The nonlinearity is due to the product \(y y'\) and the \(\sin(y)\) term.

    \item[(d)] \(\mathbf{\frac{y'}{t} + \tan(t) y = 1}\) \\
    Yes, this equation is linear. It can be rewritten in linear normal form as:
    \[
    \frac{dy}{dt} + t \tan(t) y = t
    \]

    \item[(e)] \(\mathbf{y' + \frac{t}{y} = t^2}\) \\
    No, this equation is not linear. The nonlinearity is due to the \(\frac{t}{y}\) term.

    \item[(f)] \(\mathbf{\sin(t) y' - \frac{\log(t^2) + \frac{y}{t}}{1 + e^{2t}} = y'}\) \\
    Yes, this equation is linear. It can be simplified to:
    \[
    y' \left(\sin(t) - 1\right) = \frac{\log(t^2) + \frac{y}{t}}{1 + e^{2t}}
    \]
    and then put into linear normal form.
\end{enumerate}



\textbf{Problem I.2.(4).}

Given the differential equation:

\[
\cos(x) \frac{dy}{dx} = -\sin(x) y
\]

We can rewrite it in linear normal form:

\[
\frac{dy}{dx} + \tan(x) y = 0
\]

To solve this, we first find the integrating factor \(A(x)\):

\[
A(x) = \int \tan(x) \, dx = \ln|\sec(x)|
\]

The integrating factor is then:

\[
e^{A(x)} = \sec(x)
\]

Multiplying the entire differential equation by the integrating factor:

\[
\sec(x) \frac{dy}{dx} + \sec(x) \tan(x) y = 0
\]

Recognizing that the left-hand side is the derivative of \(\sec(x) y\), we have:

\[
\frac{d}{dx} \left(\sec(x) y \right) = 0
\]

Integrating both sides with respect to \(x\):

\[
\sec(x) y = C
\]

Finally, solving for \(y(x)\):

\[
y(x) = C \cos(x)
\]

where \(C\) is the constant of integration.



\textbf{Problem I.2.(14).}

Consider the differential equation:

\[
\tan(t) \frac{dy}{dt} - y = \frac{1}{1-t}, \quad \text{with initial condition } y(2) = 1
\]

First, we rewrite it in linear normal form:

\[
\frac{dy}{dt} - \cot(t) y = \frac{\cot(t)}{1-t}
\]

Here, \(a(t) = -\cot(t)\), so we compute the integrating factor \(A(t)\):

\[
A(t) = \int -\cot(t) \, dt = -\ln|\sin(t)|
\]

The integrating factor is then:

\[
e^{A(t)} = \frac{1}{\sin(t)} = \csc(t)
\]

Multiplying the differential equation by the integrating factor:

\[
\csc(t) \frac{dy}{dt} - \csc(t) \cot(t) y = \frac{\csc(t) \cot(t)}{1-t}
\]

Recognizing that the left-hand side is the derivative of \(\csc(t) y\), we have:

\[
\frac{d}{dt} \left(\csc(t) y \right) = \frac{\csc(t) \cot(t)}{1-t}
\]

Integrating both sides with respect to \(t\):

\[
\csc(t) y = \int_{2}^{t} \frac{\csc(s) \cot(s)}{1-s} \, ds + C
\]

Substituting back to solve for \(y(t)\):

\[
y(t) = \sin(t) \left( \int_{2}^{t} \frac{\csc(s) \cot(s)}{1-s} \, ds + \frac{1}{\sin(2)} \right)
\]

where \(C = \frac{1}{\sin(2)}\) is determined by the initial condition \(y(2) = 1\).

The interval of definition for this solution is (1, $\pi$).



































\end{document}
